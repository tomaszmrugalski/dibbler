%%
%% $Id: dibbler-user-usage.tex,v 1.6 2004-12-08 00:20:57 thomson Exp $
%%
%% $Log: not supported by cvs2svn $
%% Revision 1.5  2004/10/25 20:45:54  thomson
%% Option support, parsers rewritten. ClntIfaceMgr now handles options.
%%
%% Revision 1.4  2004/07/05 00:12:30  thomson
%% Lots of minor changes.
%%
%% Revision 1.3  2004/06/19 19:51:14  thomson
%% Various fixes.
%%
%% Revision 1.2  2004/06/19 10:24:59  thomson
%% Hyperlinks in PDF, building process modified
%%

\section{Installation and usage}
Both client and server are installed in the same way. Installation
method is different in WindowsXP and Linux systems, so they're described
separately. To simplify installation, it assumes that binary versions
are used\footnote{Compilation is not
  required, binary version can be used safely. Compilation can be performed by
  advanced users, see \emph{Compilation} section for details.}.

\subsection{Linux installation}
Obtain (e.g. download from \url{http://klub.com.pl/dhcpv6/}) an archive with
Dibbler binaries and extract it to \verb+/var/lib/dibbler+ directory:
\begin{verbatim}
cd /var/lib/dibbler/
tar zxvf dibbler-0.2.1-linux.tar.gz 
\end{verbatim}

Depending what functionality do you want to use (server or client),
you should edit config file (\verb+client.conf+ for client and \verb+server.conf+
for server). After editing, issue following commands:

\begin{verbatim}
./dibbler-server start
\end{verbatim}

for starting server or

\begin{verbatim}
./dibbler-client start
\end{verbatim}

for starting client. \verb+start+ parameter needs a little comment. It
instructs Dibbler to run in daemon mode -- detach from console and run
in the background. During config files fine-tuning, it is ofter better
to watch Dibbler's bahavior instantly. In this case, use \verb+run+
instead of \verb+start+ parameter. Dibbler will present its messages on
your console. To finish it, press ctrl-c.

To stop server running in daemon mode, type:
\begin{verbatim}
./dibbler-server stop
\end{verbatim}

To stop client running in daemon mode, type:
\begin{verbatim}
./dibbler-client stop
\end{verbatim}

To see, if client or server are running\footnote{Running status is
  based on /var/lib/dibbler/client.pid or server.pid files. In rare
  occasions, when server crashes, this status will show server status as running.}, type:
\begin{verbatim}
./dibbler-client status
\end{verbatim}

or
\begin{verbatim}
./dibbler-server status
\end{verbatim}

\subsection{Windows installation}
Starting at 0.2.1, Dibbler supports Windows XP and 2003. 
Obtain (e.g. download from \url{http://klub.com.pl/dhcpv6/}) an archive with
Dibbler binaries and extract it to a directory, e.g. \verb+c:\dibbler+.

Open a console (start $\rightarrow$ run... $\rightarrow$ cmd) and
issue following commands:

\begin{verbatim}
c:
cd \dibbler
dibbler-client run -d c:\dibbler
\end{verbatim}

\verb+run+ command instructs Dibbler to run in a console and not become a
daemon. Is it very useful feature in early configuration stages. To
finish it, press ctrl-c. 

After configuration files are edited and tested, user can install Dibbler as a
Windows service. In this case, following command should be issued:
\begin{verbatim}
dibbler-client install -d c:\dibbler
\end{verbatim}

Now dibbler is installed as a Windows service and it can be controlled
from Control Panel $\rightarrow$ Administrative tasks $\rightarrow$
Services. 

If you want to uninstall Dibbler as a service, use \verb+uninstall+
instead of \verb+install+.

\section{Compilation}
Dibbler is distributed in 2 versions: binary and source files. For
most users, binary version is better choice.  Compilation is
performed by more experienced users, preferably with programming
knowledge. It does not offer any advances over binary version, only
allows to understand internal Dibbler workings. You probalby want just
install and use Dibbler. If that is your case, read section
named \emph{Installation}.

\subsection{Linux compilation}

Compilation in most cases is not necessary and should be performed
only by experienced users. Perferred method is to use binaries
provided on Dibbler's website. Issue following commands:
\begin{verbatim}
tar zxvf dibbler-0.3.0-src.tar.gz
cd dibber
make
\end{verbatim}
That's it. If there are problems with missing/different compiler
version, take a look at Makefile.inc file. Dibbler was compiled using
gcc 2.95, 3.0, 3.2, 3.3 and 3.4 versions. Lexer files were generated using
flex 2.5.31. Parser file were created using bison++
1.21.9\footnote{flex and bison++ tools are not required to compile
  Dibbler. Generated files are placed in CVS and in tar.gz
  archives}. Everything was developed under Debian GNU/Linux system.

If there are problems with \verb+SrvLexer.cpp+ and
\verb+ClntLexer.cpp+ files, please use FlexLexer.h in port-linux/
directory. Most simple way to do this is to copy this file to
\verb+/usr/include+ directory. Additional information about
compilation can be found in \emph{Dibbler Developer's Guide}.

\subsection{WindowsXP compilation}
Download dibbler-0.3.0-src.tar.gz and extract it. In port-winxp there
will be project files (one for server and one for client) for MS
Visual C++ 2003. Open one of them and click Build command. That should
do the trick. Additional information about compilation can be found in
\emph{Dibbler Developer's Guide}.
