%%
%% $Id: dibbler-user-epilogue.tex,v 1.12 2007-05-04 17:13:39 thomson Exp $
%%
%% $Log: not supported by cvs2svn $
%% Revision 1.11  2007-04-01 19:22:26  thomson
%% 0.6.0RC4 descriptions.
%%

\newpage
\section{Miscellaneous topics}

\subsection{History}
Dibbler project was started as master thesis by Tomasz Mrugalski and
Marek Senderski on Computer Science faculty on Gdansk University of
Technology. Both authors graduated in september 2003 and soon after
started their jobs.

During master thesis writing, it came to my attention that there are
other DHCPv6 implementations available, but none of them has been
named properly. Refering to them was a bit
silly: ,,DHCPv6 published on sourceforge.net has better support than
DHCPv6 developed in KAME project, but our DHCPv6
implementation...''. So I have decided that this implementation should
have a name. Soon it was named Dibbler after famous CMOT
Dibbler from Discworld series by Terry Pratchett.

Sadly, Marek does not have enough free time to develop Dibbler, so his
involvement is non-existent at this time. However, that does not mean,
that this project is abandoned. It is being actively developed by
me (Tomek). Keep in mind that I work at full time and do
Ph.D. studies, so my free time is also greatly limited.

\hypertarget{contact}{}
\subsection{Contact and reporting bugs}
\label{contact}
There is an website located at \url{http://klub.com.pl/dhcpv6}. If you
belive you have found a bug, please put it in Bugzilla -- it is a bug
tracking system located at \url{http://klub.com.pl/bugzilla}. If you
are not familiar with that kind of system, don't worry. After simple
registration, you will be asked for system and Dibbler version you are
using and other details. Without feedback from users, author will not
be aware of many bugs and so will not be able to fix them. That's why
users feedback is very important. You can also send bug report to the
mailing list if you don't want to use bugzilla directly (or want to
confirm first that it is indeed a bug). Be sure to be as detailed as
possible. Please include both server and client log files, both config
and xml files. If you are familiar with tcpdump or ethereal, traffic
dumps from this programs are also great help.

If you are not sure if your issue is a bug or a configuration problem,
you may also want to browse archives and ask on a mailing list. See
following subsection for details.

If you have used Dibbler and it worked ok, this documentation answered
all you question and everything is in order (hmmm, wake up, it must be
a dream, it isn't reality:), also send a short note to the mailing
list.

Author keeps a list of places where Dibbler software is used. He would
appreciate if you could check if your country is on the list (see
project website) and mention it if it isn't. That's completely
optional and author won't be disappointed if you chose to not reveal
that information.

Finally, while the author's mail isn't secret, please \emph{DO NOT}
send mails to him directly. He is quite busy and do not want to
respond to the same questions over and over again. Also, he travels a
lot, so often is unable to respons. It is much better to ask on the
mailing list, which has public archive searchable by Google. This
could help other people who may have the same question. And even if
they ask the question without bothering to google for answers first,
it will be easier for the author to respond with a link to previous
response. Finally, there's currently over 100 people subscribed to the
list, so there's a non-trivial chance that some of them will respond
when author is not available.

Please constact author directly \emph{ONLY} if you want to report
security issue or want to discuss confidential matters.

\subsection{Mailing lists}
\label{mailing-list}
There are two mailing lists related to the Dibbler project:
\begin{description}
\item[dibbler] -- Maling list for Dibbler users. It is used to ask for help,
report bugs, hay hello and things like that. If you are not sure, what to
do, people on this list will try to help you. Web-inteface link:
\href{http://klub.com.pl/cgi-bin/mailman/listinfo/dibbler}{http://klub.com.pl/cgi-bin/mailman/listinfo/dibbler}
\item[dibbler-devel] -- That list is intended as a way of communication
between people, who are technically involved in the dibbler development.
If you are going to improve dibbler in any way, make sure that you announce
it here. You may get help. Also if you are trying to fix a bug on your own
(hey, that's great!), this list is a good place to talk about it.
Web-interface link: \href{http://klub.com.pl/cgi-bin/mailman/listinfo/dibbler-devel}{http://klub.com.pl/cgi-bin/mailman/listinfo/dibbler-devel}
\end{description}

Both lists have archives available on-line. You can join or leave one or both lists
at any time using convenient web-interface or using traditional mail-based approach.

\subsection{Thanks and greetings}

I would like to send my thanks and greetings to various persons.
Without them, Dibbler would not be where it is today. For a full list
of contributors, see AUTHORS file.

\begin{description}
\item[Marek Senderski] -- He's author of almost half of the Dibbler
  code. Without his efforts, Dibbler would be simple, long forgotten
  by now master thesis.
\item[Jozef Wozniak] -- My master thesis' supervisor. He allowed me to
  see DHCP in a larger scope as part of network provisioning process.
\item[Jacek Swiatowiak] -- He's my master thesis consultant. He guided
  Marek and me to take first steps with DHCPv6 implementation.
\item[Ania Szulc] -- Discworld fan and a great girl, too. She's the one
  who helped me to decide how to name this yet-untitled DHCPv6 implementation.
\item[Christian Strauf] -- Without his queries and questions, Dibbler
  would be abandoned in late 2003.
\item[Bartek Gajda] -- His interest convinced me that Dibbler is worth
  the effort to develop it further.
\item[Artur Binczewski and Maciej Patelczyk] -- They both ensured that
  Dibbler is (and always will be) GNU GPL software. Open source
  community is grateful.
\item[Josep Sole] -- His mails (directly and indirectly) resulted in
  various fixes and speeded up of 0.2.0 release.
\item[Sob] -- He has ported 0.4.0 back to Win2000 and NT. As a direct
  result, 0.4.1 was released for those platforms, too.
\item[Guy ''GMSoft'' Martin] -- He has provided me with access to HPPA
  machine, so I was able to squish some little/big endian bugs. He
  also uploaded ebuild to the Gentoo portage.
\item[Bartosz ''fEnio'' Fenski] -- He taught me how much work needs to
  be done, before deb packages are considered ok. It took me some time
  to understand that more pain for the package developer means less
  problems for the end user.  Thanks to him, Dibbler is now part of
  the Debian GNU/Linux distribution.
\item[Adrien Clerc and his team] -- Their contribution of the DNS
  Updates code is most welcome.
\item[Krzysztof Wnuk] -- He has fixed, improved and extended DNS
  Updates support as well as provided initial support for prefix
  delegation.
\item[Alain Durand] -- Thanks for the invitation to interop test
  session and for allowing me to see DHCPv6 issues in a much broader
  scope.
\item[Petr Pisar] -- He has reported lots of bugs, and also often
  provides fixes.  Thanks.
\item[Paul Schauer] -- Thanks to his effors, Dibbler now works on Mac
  OS X. He did majority of the porting work and then did numerous
  rounds of testing and debugging.
\end{description}

\newpage
\section{Acknowledgements}
Author would like to acknowledge following projects and programmes
that supported or continue to support research and development of
the Dibbler software and related activities.

\vspace{1cm}

\noindent
This work has been partially supported by the Polish Ministry of
Science and Higher Education under the European Regional Development
Fund, Grant No. POIG.01.01.02-00-045/09-00 
\href{https://www.iip.net.pl/en/project}{Future Internet Engineering.}

\begin{figure}[ht]
\begin{minipage}[b]{0.27\textwidth}
\centering
\vspace*{\fill}
\includegraphics[scale=0.45]{logo-nss}
\vspace*{\fill}
\end{minipage}
\begin{minipage}[b]{0.28\textwidth}
\centering
\includegraphics[scale=0.7]{logo-iip}
\end{minipage}
\begin{minipage}[b]{0.16\textwidth}
\centering
\includegraphics[scale=0.6]{logo-pg}
\end{minipage}
\begin{minipage}[b]{0.26\textwidth}
\centering
\includegraphics[scale=0.75]{logo-eu}
\end{minipage}
\end{figure}

\vspace{1cm}

\noindent
The Dibbler project was created as master thesis at 
\href{http://www.eti.pg.gda.pl/lang?locale=en/}{Department
of Computer Communications}, at \href{http://www.eti.pg.gda.pl/lang?locale=en}{Faculty of Electronics,
Telecommunications and Informations}, at \href{http://www.pg.gda.pl/en/}{Gdansk University of Technology}.

\begin{figure}[ht]
\begin{minipage}[b]{0.33\linewidth}
\centering
\includegraphics[scale=1.0]{logo-eti}
\vspace{-0.8cm}
\end{minipage}
\begin{minipage}[b]{0.33\linewidth}
\centering
\includegraphics[scale=1.1]{logo-pg}
\end{minipage}
\hspace{0.5cm}
\begin{minipage}[b]{0.33\linewidth}
\centering
\includegraphics[scale=0.6]{logo-kti}
\end{minipage}
\end{figure}

\newpage
