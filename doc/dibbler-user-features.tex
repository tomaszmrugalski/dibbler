\section{How to setup features}
This recently added section will contain information about setting up
various Dibbler features.

\subsection{DNS Update}
During normal operation, DHCPv6 client receives one or more IPv6 address(es)
from DHCPv6 server. If configured to do so, it can also receive
information about DNS server addresses. As an additional service, DNS
Update can be performed. This feature, sometimes known as Dynamic DNS,
keeps DNS entries up to date. When client boots, it gets its fully
qualified domain name and this name can be used to reach this
particular client.

There are two types of the DNS Updates. First is a so called forward
resolving. It allows to change a node's name into its address,
e.g. malcolm.example.com can be translated into 2000::123. Other kind
of record, which can be updated is a so called reverse resolving. It
allows to obtain full name of a node with know address, e.g. 2000::124
can be translated into zoe.example.com.

To configure this feature, following steps must be executed:

\begin{enumerate}
\item Configure DNS server. DNS server supporting IPv6 and dynamic
  updates must be configured. One example of such server is a BIND
  9.3. It is necessary to allow listening on the IPv6 sockets and
  define that specific domain can be updated. See example below.
\item Configure Dibbler server to provide DNS server informations for
  clients. DNS Updates will be sent to the first DNS server on the
  list of available servers.
\item Configure Dibbler server to work in stateful mode, i.e. that it
  can provide addresses for the clients. This is a default mode, so
  unless configuration was altered, this step is already done. Make
  sure that there is no ,,stateless'' keyword in the
  \verb+server.conf+ file.
\item Define list of the available names in the server configuration
  file. Make sure to use fully qualified domain names
  (e.g. malcolm.example.com), not the hostnames only. 
\item Configure dibbler client to request for DNS Update. Use ,,option
  fqdn'' to achieve this. 
\end{enumerate}
  
\begin{verbatim}
options {
        listen-on-v6 { any; };
        listen-on { any; };

        // other options here
        // ...
};

zone "example.com" {
          type master;
          file "pri/example.com";
          allow-update { any; };
          allow-transfer { any; };
          allow-query { "any"; };
          notify yes;

      // other options follow
      // ...
};

\end{verbatim}

Also see example server and client configuration files described in a
later sections of this document. Also note that Dibbler distribution
should be accompanied with several example configuration files. Some
of them include FQDN usage examples.

TODO: Provide more details.